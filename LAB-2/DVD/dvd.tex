\documentclass[10pt,a4paper]{article}
\usepackage[utf8]{inputenc}
\usepackage[T1]{fontenc}
\usepackage[spanish]{babel}
\usepackage{amsmath}
\usepackage{amsfonts}
\usepackage{amssymb}
\usepackage{graphicx}
\usepackage{float}
\usepackage{multicol}
\usepackage[dvipsnames]{xcolor}
\usepackage{hyperref}
\definecolor{pinegreen}{rgb}{0.36, 0.54, 0.66}
\hypersetup{
    colorlinks=true,
    linkcolor=pinegreen,      
    urlcolor=red,
    }
\addto{\captionsspanish}{\renewcommand{\abstractname}{Abstract}}
\newcommand{\celda}[1]{
	\begin{minipage}{2cm}
		\vspace{2mm}
		#1
		\vspace{2mm}
	\end{minipage}
}
\definecolor{pinegreen}{rgb}{0.36, 0.54, 0.66}
\usepackage[left=2.00cm, right=2.00cm, top=2.00cm, bottom=2.00cm]{geometry}
%-----------------------
\usepackage{fancyhdr}
\usepackage{lastpage}   
\pagestyle{fancy} 
\fancyhf{}   
\rhead{Gabriel Hernández}                       %Create right header text
\chead{Medición de la capacidad de un DVD}  
\lhead{Laboratorio 2}                        %Create left header text
                   %Create center header text

\rfoot{Page \thepage \hspace{1pt} of \pageref{LastPage}}    %Create right footer text
\renewcommand{\headrulewidth}{0.2pt}      %Set header line thickness to 0 pt [default=1pt]
\renewcommand{\footrulewidth}{0pt}      %Set footer line thickness to 0 pt [default =0pt]
%-----------------------
\author{Gabriel Hernandez Bello}
\begin{document}
	
	\begin{figure}[H]
		\raggedright
		\includegraphics[scale=0.2]{IMG/logo_udec.png} \hfill \includegraphics[scale=0.5]{IMG/cfm_logo.png}
	\end{figure}

	\vspace{6mm}
	%ESTE CENTER ES EXCLUSIVO PARA EL TITULO DEL PAPER, AUTOR Y UNIVER.
	\begin{center}
		{\Large \textbf{Medición de la capacidad de un DVD}}\\
		\vspace{2mm}
		{\large Gabriel Hernández Bello$^{1}$}\\
		\vspace{6.5mm}
		$^1$\textit{Universidad de Concepción, Facultad de Ciencias Físicas y Matemáticas, Ciencias Físicas. }\\
	\end{center}

	\begin{center}
		\textcolor{pinegreen}{\rule{150mm}{0.8mm}}
	\end{center}

      %ESTE ABSTRACT ES PARA EL RESUMEN PROPIAMENTE DICHO Y PARA LAS PALABRAS CLAVES (KEYWORDS) ,NOTA:el comando \par sirve para iniciar el nuevo parrafo con sangría.
	\begin{abstract}
	A partir de las mediciones de distancias y ángulos calculadas con una herramienta de propia fabricación se obtendrá una estimación de la altura del Campanil de la Universidad de Concepción.\\
	\\
		\textbf{Palabras Claves ---}  Campanil, Trigonometría, Aproximación Lineal.
	\end{abstract}
	
	\begin{center}
		\textcolor{pinegreen}{\rule{150mm}{0.8mm}}
	\end{center}
	
	\begin{multicols}{2}
		\section{Introducción}
			El DVD es un disco óptico para almacenamiento de datos. Las siglas DVD corresponden a \emph{Disco Versátil Digital}. Fue creado y desarrolloado en 1995 con su primer lanzamiento para 1 Noviembre de 1996, en Japón. El medio puede almacenar cualquier tipo de datos digitales y se ha utilizado ampliamente para almacenar programas de vídeo (vistos con reproductores de DVD), software y otros archivos informáticos. Los DVD ofrecen una capacidad de almacenamiento significativamente mayor que los discos compactos (CD) aunque tienen las mismas dimensiones. Un DVD estándar de una sola capa puede almacenar hasta 4,7 GB de datos, un DVD de doble capa hasta 8,5 GB. Las variantes pueden almacenar hasta un máximo de 17,08 GB \cite{wikiDVD}.\\
			
			En este laboratorio aprovecharemos las caracterísitcas ópticas del DVD para estimar su capacidad.
		\section{Marco Teórico}
		\subsection{Especificaciones del DVD}
		Los DVD almacenan información en una línea de \emph{surcos} microescópicos dispuestos en una espiral que va desde el centro del disco hacia la circunsferencia exterior. Así, todos los lectores de DVD utilizan láseres para leer la información codificada en los surcos. La longitud de onda del láser depende de la separación de los surcos, en general esta es de 0.74 [$\mu m$].\\
		
		Los DVD constan de tres capas: el sustrato de policarbonato transparente, la capa de colorante y la capa reflectante. En partícular, la capa de reflexión se encuentra en medio de las capas de policarbonato. La más importante es la capa de reflexión que consta, en la mayoría de los casos, de una lámina de aluminio que vuelve la superficie del DVD relfectante. Lo anterior permite que el láser del lector se refleje para ser leído por el sensor de recogida de la unidad fotocaptora. \cite{especificaciones_DVD}.
		\subsection{Trigonometría}
		Etimológicamente la palabra trigonometría significa \textit{medida de los triángulos}. En efecto, es la ciencia que estudia las relaciones que ligan los lados de un triángulo y aplican esas realciones al cálculo de los elementos desconocidos. En particular, la trigonometría se basa en el conocimiento de los ángulos de un triángulo que establecen conexiones llamadas \textit{relaciones trigonométricas}. Las principales son el \emph{seno, coseno} y la \emph{tangente} \cite{trigonometria}.
		\section{Procedimiento Experimental y Resultados}
		Para este experimento utilizamos un DVD usual, compuesto por una capa vacía y otra con información (en adelante \emph{capa/disco óptico}), luego separamos ambas capas del DVD para trabajar únicamente con el disco que contiene toda la infomación. De esta forma, colocamos una pantalla de observación atrás del disco óptico y apuntamos un láser hacia este generando un patrón de difracción, visible en la pantalla, a causa de los surcos que componen la capa óptica. Para asegurar la estabilización del láser utilizamos un soporte universal.\\
		
		Por su parte, medimos experimentalmente la distancia entre el láser y la pantalla de observación ($D$), la distancia entre la separación de los surcos observado en el patrón de difracción ($l$) y luego calculamos el ángulo de apertura ($\theta$) para la posterior estimación de la separación entre los surcos en el disco óptico.
	\end{multicols}
	
	\begin{table}[H]
		\centering
		\begin{tabular}{|c|c|c|c|}
			\hline
			$D$ (m) & $L$ (m) & $\theta$ (rad) )&  $d$(nm) \\ \hline
			0.0240 $\pm$ 0.0005  & 0.034 $\pm$ 0.0005 &  0.956  & 783.407 \\
			0.0300 $\pm$ 0.0005  & 0.0440 $\pm$ 0.0005 &  0.97  & 757.865\\
			0.0400 $\pm$ 0.0005  & 0.0610 $\pm$ 0.0005 &  0.99  & 765.526\\ 
			0.0500 $\pm$ 0.0005 & 0.0730 $\pm$ 0.0005 & 0.97  & 775.865\\ 
			0.0600 $\pm$ 0.0005 & 0.0890 $\pm$ 0.0005 & 0.977 &  772.179\\ 
			0.0700 $\pm$ 0.0005 & 0.1030 $\pm$ 0.0005 & 0.973  & 774.27 \\
			0.0800 $\pm$ 0.0005  & 0.1190 $\pm$ 0.0005 &  0.978 & 771.659 \\ 
			0.0900 $\pm$ 0.0005 & 0.1360 $\pm$ 0.0005 &  0.986  & 767.547 \\
			0.1000 $\pm$ 0.0005 & 0.1550 $\pm$ 0.0005 &  0.997 & 762.044  \\
			0.1100 $\pm$ 0.0005 & 0.1640 $\pm$ 0.0005 &  0.979  & 771.140 \\
			0.1200 $\pm$ 0.0005 & 0.1720 $\pm$ 0.0005 &  0.961  & 780.711  \\
			0.1300 $\pm$ 0.0005 & 0.1860 $\pm$ 0.0005 &  0.960 & 781.258 \\
			0.1400 $\pm$ 0.0005 & 0.2030 $\pm$ 0.0005 &  0.967 & 777,466  \\
			0.1500 $\pm$ 0.0005 & 0.2200 $\pm$ 0.0005 &  0.972 & 774.804  \\ \hline
			
		\end{tabular}
		\caption{Ángulos y Distancias Medidas para el láser rojo}
		\label{tab:angulos_distancias.}
		\rule{100mm}{0.1mm}
	\end{table}
	
	\begin{table}[H]
		\centering
		\begin{tabular}{|c|c|c|c|}
			\hline
			$D$ (m) & $L$ (m) & $\theta$ (rad) )&  $d$(nm) \\ \hline
			0.0220 $\pm$ 0.0005  & 0.0240 $\pm$ 0.0005 &  0.825  & 0 \\
			0.0300 $\pm$ 0.0005  & 0.0320 $\pm$ 0.0005 &  0.817  & 0\\
			0.0400 $\pm$ 0.0005  & 0.0420 $\pm$ 0.0005 &  0.809  & 0\\ 
			0.0500 $\pm$ 0.0005 & 0.0570 $\pm$ 0.0005 & 0.850  & 0\\ 
			0.0600 $\pm$ 0.0005 & 0.0660 $\pm$ 0.0005 & 0.832 &  0\\ 
			0.0700 $\pm$ 0.0005 & 0.0770 $\pm$ 0.0005 & 0.832  & 0 \\
			0.0800 $\pm$ 0.0005  & 0.0850 $\pm$ 0.0005 &  0.815 & 0 \\ 
			0.0900 $\pm$ 0.0005 & 0.0980 $\pm$ 0.0005 &  0.827  & 0 \\
			0.1000 $\pm$ 0.0005 & 0.1090 $\pm$ 0.0005 &  0.828 & 0  \\
			0.1100 $\pm$ 0.0005 & 0.1200 $\pm$ 0.0005 &  0.828  & 0 \\
			0.1200 $\pm$ 0.0005 & 0.1310 $\pm$ 0.0005 &  0.829  & 0  \\
			0.1300 $\pm$ 0.0005 & 0.1430 $\pm$ 0.0005 &  0.832 & 0 \\
			0.1400 $\pm$ 0.0005 & 0.1510 $\pm$ 0.0005 &  0.823 & 0  \\
			0.1500 $\pm$ 0.0005 & 0.1610 $\pm$ 0.0005 &  0.820 & 0  \\ \hline
			
		\end{tabular}
		\caption{Ángulos y Distancias Medidas para el láser verde}
		\label{tab:angulos_distancias.}
		\rule{100mm}{0.1mm}
	\end{table}

	\begin{multicols}{2}
	\section{Análisis}
Los gráficos \ref{Grafico campanil} y \ref{Grafico campanil ajustado}  presentan los datos obtenidos durante el proceso experimental junto a su ajuste lineal, 		representado por la línea punteada. Del gráfico \ref{Grafico campanil} notamos que los datos se encuentran muy próximos entre si, 	con una desviación estándar de solo 0.3. Además, el ajuste lineal de los datos da como resultado una línea recta con una pendiente, aproximadamente nula, de 0.001 [m/rad]. Debido a esto, decidimos visualizar los datos en una escala más apropiada. Así, en el gráfico \ref{Grafico campanil ajustado} la aproximación lineal muestra una notoria linea recta que estabiliza el valor de la altura del Campanil a 41.64 [m]. Lo anterior es de esperar, pues estamos comparando la altura del campanil para distintos valores del ángulo $\alpha$. Cabe destacar que se seleccionaron solo siete datos, pues son suficientes para ajustar una recta que pase por todos los puntos y sus respectivos errores.\\

Paralelamente, observamos que según la ecuación \ref{Ec. altura} el valor $D - d = L \tan(\alpha)$ debe ser una constante, pues la altura del Campanil $D$ es una constante (teóricamente) y no variamos  $d$ durante el proceso experimental. Esto nos permite validar el comportamiento de los datos verificando que las variables $L$ y $\tan(\alpha)$ exhiben una relación inversamente proporcional. En efecto, en el gráfico \ref{Grafico de proporcionalidad} evidenciamos una relación lineal inversamente proporcional. Luego, concluimos que los datos obtenidos son coherentes.\\
\\
\\
\\
En definitiva, considerando el valor al que converge la altura del Campanil según la aproximación lineal de los datos y el error asociado a este parámetro, estimamos la altura del Campanil $D$ en este experimento como: $$D = 41.64 \pm 1.31 [m].$$
	

Ahora bien, la altura del Campanil es un dato conocido. Con ello, es posible calcular el error absoluto y porcentual de nuestra estimación, de esta forma:
\begin{align*}
\epsilon_{abs} &= |42.5[m] - 41.64[m]| = 0.86 [m]\\
\epsilon_{rel} &= \frac{0.86[m]}{41.64[m]}\cdot 100 = 2.06 \% 
\end{align*}
Mostrando así la validez de la estimación realizada en este experimento.
	
\section{Conlusión}
En el análisis, demostramos la efectividad del método empleado para estimar la altura del campanil, viéndose reflejado en el bajo error absoluto y relativo de la medición. Sin embargo, observamos que el error asociado a la estimación de la altura del Campanil tiene un valor relevante de $\pm$ 1.31 [m], lo cual atribuimos a la falta de precisión en la medición durante el proceso experimental. A pesar de ello, los resultados obtenidos se acercan considerablemente al valor esperado, lo que destaca el tratamiento eficaz en el análisis de los datos junto a la gran capacidad y utilidad de las herramientas matemáticas utilizadas en la estimación de la altura del Campanil.



	\bibliographystyle{unsrt}
	\bibliography{referencias}
	
	\end{multicols}
\end{document}