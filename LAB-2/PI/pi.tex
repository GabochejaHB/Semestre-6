\documentclass[10pt,a4paper]{article}
\usepackage[utf8]{inputenc}
\usepackage[T1]{fontenc}
\usepackage[spanish]{babel}
\usepackage{amsmath}
\usepackage{amsfonts}
\usepackage{amssymb}
\usepackage{graphicx}
\usepackage{float}
\usepackage{multicol}
\usepackage[dvipsnames]{xcolor}
\usepackage{hyperref}
\definecolor{pinegreen}{rgb}{0.36, 0.54, 0.66}
\hypersetup{
    colorlinks=true,
    linkcolor=pinegreen,      
    urlcolor=red,
    }
\addto{\captionsspanish}{\renewcommand{\abstractname}{Abstract}}
\newcommand{\celda}[1]{
	\begin{minipage}{2cm}
		\vspace{2mm}
		#1
		\vspace{2mm}
	\end{minipage}
}
\definecolor{pinegreen}{rgb}{0.36, 0.54, 0.66}
\usepackage[left=2.00cm, right=2.00cm, top=2.00cm, bottom=2.00cm]{geometry}
%-----------------------
\usepackage{fancyhdr}
\usepackage{lastpage}   
\pagestyle{fancy} 
\fancyhf{}   
\rhead{Gabriel Hernández}                       %Create right header text
\chead{Estimación experimental de $\pi$}  
\lhead{Laboratorio 2}                        %Create left header text
                   %Create center header text

\rfoot{Page \thepage \hspace{1pt} of \pageref{LastPage}}    %Create right footer text
\renewcommand{\headrulewidth}{0.2pt}      %Set header line thickness to 0 pt [default=1pt]
\renewcommand{\footrulewidth}{0pt}      %Set footer line thickness to 0 pt [default =0pt]
%-----------------------
\author{Gabriel Hernandez Bello}
\begin{document}
	
	\begin{figure}[H]
		\raggedright
		\includegraphics[scale=0.2]{../Altura-Campanil/IMG/logo_udec.png} \hfill \includegraphics[scale=0.5]{../Altura-Campanil/IMG/cfm_logo.png}
	\end{figure}

	\vspace{6mm}
	%ESTE CENTER ES EXCLUSIVO PARA EL TITULO DEL PAPER, AUTOR Y UNIVER.
	\begin{center}
		{\Large \textbf{Estimación experimental de \textbf{$\pi$}}}\\
		\vspace{2mm}
		{\large Gabriel Hernández Bello$^{1}$}\\
		\vspace{6.5mm}
		$^1$\textit{Universidad de Concepción, Facultad de Ciencias Físicas y Matemáticas, Ciencias Físicas. }\\
	\end{center}

	\begin{center}
		\textcolor{pinegreen}{\rule{150mm}{0.8mm}}
	\end{center}

      %ESTE ABSTRACT ES PARA EL RESUMEN PROPIAMENTE DICHO Y PARA LAS PALABRAS CLAVES (KEYWORDS) ,NOTA:el comando \par sirve para iniciar el nuevo parrafo con sangría.
	\begin{abstract}

		\textbf{Palabras Claves ---}  DVD, Láser, Patrón de Difracción, Óptica.
	\end{abstract}
	
	\begin{center}
		\textcolor{pinegreen}{\rule{150mm}{0.8mm}}
	\end{center}
	
	\begin{multicols}{2}
		\section{Introducción}
			El número $\pi$ es una de las constantes matemáticas por autonomasia. Su omnipresencia en diferentes ámbitos relacionados con la física, las matemáticas y la ingeniería lo hacen reconocible hasta para aquellos que viven alejados de las ramas científicas.\\
			
			El emblemático número irracional $\pi$ se define como la razón entre el perímetro de una circunsferencia y su diámetro. Toda investigación que incluya alguna variable relacionada con círculos, circunferencias o similares llevará implícito su cálculo, desde las elipses de las trayectorias espaciales hasta la fabricación de ruedas o balones de fútbol. A partir de ahí, su utilidad es casi tan dilatada como su número de decimales \cite{discovery}.\\
			
			En el presente laboratorio, nos concentraremos en estimar experimentalmente el valor del número $\pi$ a través de dos experimentos simples.
		\section{Marco Teórico}
		\subsection*{Definición de $\pi$}
		Sea una circunsferencia con perímetro $P$ y diámetro $D$. Se define el número $\pi$ como la razón entre estas magnitudes, tal que:
		\begin{equation}\label{def de pi}
		\pi = \frac{P}{D}
		\end{equation}
	
		\subsection*{Integrales en la antiguedad}
		Arquímedes de Siracusa (287-212 a.C.) utilizó el método exhaustivo de Euxodo de Cnido. inscribiendo y 	
circunscribiendo polígonos regulares en una circunferencia para calcular áreas y volúmenes. Calculó el volumen y la superficie de una esfera y de un cono y la superficie de una elipse y una parábola y expuso un método para calcular los volúmenes de revolución de segmentos de elipsoides, paraboloides e hiperboloides cortados por un
plano perpendicular al eje principal \cite{integrales}. \\

		De esta forma se calculaban las integrales, relacionadas a áreas y volúmenes, en la antiguedad. Esta noción es de suma importancia para el laboratorio, pues en ella esta basada una de los experimentos propuestos para la estimación de $\pi$.
		\subsection*{  }
		Consideremos un material con densidad de masa $\rho$ constante, de forma que $m = \rho V$; donde $m$ representa la masa del material y  $V$ su volumen. Ahora, recortemos una circunsferencia de radio $a$ y un cuadrado de lado $a$. Así, el área de la circunsferencia será $A = \pi a^2$ y para el cuadrado tendremos un área de $B = a^2$. Luego, vemos que:
		\begin{align*}
		\frac{\rho_A}{\rho_B} &= \frac{\frac{m}{V_A}}{\frac{m}{V_b}}, \\
		 \frac{V_A}{V_B} &= \frac{A \cdot a}{B \cdot a}.
		\end{align*}
		Donde usamos la relación $V = a \cdot A$, para una longitud $a$ y un volumen $A$. \\
		
		Finalmente, notemos que podemos simplificar las longitudes $a$ y reemlazando el valor del área $A$ y $B$ obtenemos:
		\begin{equation}\label{Relación de las áreas}
		\frac{A}{B} = \frac{\pi a^2}{a^2} = pi 
		\end{equation}
		\section{Procedimiento Experimental y Resultados}
		
		
	
	
	
		
		
	

	
\section{Conlusión}





	\bibliographystyle{unsrt}
	\bibliography{referencias}
	
	\end{multicols}
\end{document}